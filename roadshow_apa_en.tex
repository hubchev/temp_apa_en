\documentclass[
  stu,
  floatsintext,
  longtable,
  a4paper,
  nolmodern,
  notxfonts,
  notimes,
  colorlinks=true,linkcolor=black,citecolor=black,urlcolor=black]{apa7}

\usepackage{amsmath}
\usepackage{amssymb}



\geometry{right=2.5cm,left=4.5cm,top=2.5cm,bottom=2.5cm}
\fancyhfoffset[LE,RO]{0cm}

\usepackage[bidi=default]{babel}
\babelprovide[main,import]{english}


% get rid of language-specific shorthands (see #6817):
\let\LanguageShortHands\languageshorthands
\def\languageshorthands#1{}

\RequirePackage{longtable}
\RequirePackage{threeparttablex}

\makeatletter
\renewcommand{\paragraph}{\@startsection{paragraph}{4}{\parindent}%
	{0\baselineskip \@plus 0.2ex \@minus 0.2ex}%
	{-.5em}%
	{\normalfont\normalsize\bfseries\typesectitle}}

\renewcommand{\subparagraph}[1]{\@startsection{subparagraph}{5}{0.5em}%
	{0\baselineskip \@plus 0.2ex \@minus 0.2ex}%
	{-\z@\relax}%
	{\normalfont\normalsize\bfseries\itshape\hspace{\parindent}{#1}\textit{\addperi}}{\relax}}
\makeatother




\usepackage{longtable, booktabs, multirow, multicol, colortbl, hhline, caption, array, float, xpatch}
\usepackage{subcaption}
\usepackage{ltcaption}


\renewcommand\thesubfigure{\Alph{subfigure}}
\setcounter{topnumber}{2}
\setcounter{bottomnumber}{2}
\setcounter{totalnumber}{4}
\renewcommand{\topfraction}{0.85}
\renewcommand{\bottomfraction}{0.85}
\renewcommand{\textfraction}{0.15}
\renewcommand{\floatpagefraction}{0.7}

\usepackage{tcolorbox}
\tcbuselibrary{listings,theorems, breakable, skins}
\usepackage{fontawesome5}

\definecolor{quarto-callout-color}{HTML}{909090}
\definecolor{quarto-callout-note-color}{HTML}{0758E5}
\definecolor{quarto-callout-important-color}{HTML}{CC1914}
\definecolor{quarto-callout-warning-color}{HTML}{EB9113}
\definecolor{quarto-callout-tip-color}{HTML}{00A047}
\definecolor{quarto-callout-caution-color}{HTML}{FC5300}
\definecolor{quarto-callout-color-frame}{HTML}{ACACAC}
\definecolor{quarto-callout-note-color-frame}{HTML}{4582EC}
\definecolor{quarto-callout-important-color-frame}{HTML}{D9534F}
\definecolor{quarto-callout-warning-color-frame}{HTML}{F0AD4E}
\definecolor{quarto-callout-tip-color-frame}{HTML}{02B875}
\definecolor{quarto-callout-caution-color-frame}{HTML}{FD7E14}

%\newlength\Oldarrayrulewidth
%\newlength\Oldtabcolsep


\usepackage{hyperref}



\usepackage{color}
\usepackage{fancyvrb}
\newcommand{\VerbBar}{|}
\newcommand{\VERB}{\Verb[commandchars=\\\{\}]}
\DefineVerbatimEnvironment{Highlighting}{Verbatim}{commandchars=\\\{\}}
% Add ',fontsize=\small' for more characters per line
\usepackage{framed}
\definecolor{shadecolor}{RGB}{241,243,245}
\newenvironment{Shaded}{\begin{snugshade}}{\end{snugshade}}
\newcommand{\AlertTok}[1]{\textcolor[rgb]{0.68,0.00,0.00}{#1}}
\newcommand{\AnnotationTok}[1]{\textcolor[rgb]{0.37,0.37,0.37}{#1}}
\newcommand{\AttributeTok}[1]{\textcolor[rgb]{0.40,0.45,0.13}{#1}}
\newcommand{\BaseNTok}[1]{\textcolor[rgb]{0.68,0.00,0.00}{#1}}
\newcommand{\BuiltInTok}[1]{\textcolor[rgb]{0.00,0.23,0.31}{#1}}
\newcommand{\CharTok}[1]{\textcolor[rgb]{0.13,0.47,0.30}{#1}}
\newcommand{\CommentTok}[1]{\textcolor[rgb]{0.37,0.37,0.37}{#1}}
\newcommand{\CommentVarTok}[1]{\textcolor[rgb]{0.37,0.37,0.37}{\textit{#1}}}
\newcommand{\ConstantTok}[1]{\textcolor[rgb]{0.56,0.35,0.01}{#1}}
\newcommand{\ControlFlowTok}[1]{\textcolor[rgb]{0.00,0.23,0.31}{\textbf{#1}}}
\newcommand{\DataTypeTok}[1]{\textcolor[rgb]{0.68,0.00,0.00}{#1}}
\newcommand{\DecValTok}[1]{\textcolor[rgb]{0.68,0.00,0.00}{#1}}
\newcommand{\DocumentationTok}[1]{\textcolor[rgb]{0.37,0.37,0.37}{\textit{#1}}}
\newcommand{\ErrorTok}[1]{\textcolor[rgb]{0.68,0.00,0.00}{#1}}
\newcommand{\ExtensionTok}[1]{\textcolor[rgb]{0.00,0.23,0.31}{#1}}
\newcommand{\FloatTok}[1]{\textcolor[rgb]{0.68,0.00,0.00}{#1}}
\newcommand{\FunctionTok}[1]{\textcolor[rgb]{0.28,0.35,0.67}{#1}}
\newcommand{\ImportTok}[1]{\textcolor[rgb]{0.00,0.46,0.62}{#1}}
\newcommand{\InformationTok}[1]{\textcolor[rgb]{0.37,0.37,0.37}{#1}}
\newcommand{\KeywordTok}[1]{\textcolor[rgb]{0.00,0.23,0.31}{\textbf{#1}}}
\newcommand{\NormalTok}[1]{\textcolor[rgb]{0.00,0.23,0.31}{#1}}
\newcommand{\OperatorTok}[1]{\textcolor[rgb]{0.37,0.37,0.37}{#1}}
\newcommand{\OtherTok}[1]{\textcolor[rgb]{0.00,0.23,0.31}{#1}}
\newcommand{\PreprocessorTok}[1]{\textcolor[rgb]{0.68,0.00,0.00}{#1}}
\newcommand{\RegionMarkerTok}[1]{\textcolor[rgb]{0.00,0.23,0.31}{#1}}
\newcommand{\SpecialCharTok}[1]{\textcolor[rgb]{0.37,0.37,0.37}{#1}}
\newcommand{\SpecialStringTok}[1]{\textcolor[rgb]{0.13,0.47,0.30}{#1}}
\newcommand{\StringTok}[1]{\textcolor[rgb]{0.13,0.47,0.30}{#1}}
\newcommand{\VariableTok}[1]{\textcolor[rgb]{0.07,0.07,0.07}{#1}}
\newcommand{\VerbatimStringTok}[1]{\textcolor[rgb]{0.13,0.47,0.30}{#1}}
\newcommand{\WarningTok}[1]{\textcolor[rgb]{0.37,0.37,0.37}{\textit{#1}}}

\providecommand{\tightlist}{%
  \setlength{\itemsep}{0pt}\setlength{\parskip}{0pt}}
\usepackage{longtable,booktabs,array}
\usepackage{calc} % for calculating minipage widths
% Correct order of tables after \paragraph or \subparagraph
\usepackage{etoolbox}
\makeatletter
\patchcmd\longtable{\par}{\if@noskipsec\mbox{}\fi\par}{}{}
\makeatother
% Allow footnotes in longtable head/foot
\IfFileExists{footnotehyper.sty}{\usepackage{footnotehyper}}{\usepackage{footnote}}
\makesavenoteenv{longtable}

\usepackage{graphicx}
\makeatletter
\newsavebox\pandoc@box
\newcommand*\pandocbounded[1]{% scales image to fit in text height/width
  \sbox\pandoc@box{#1}%
  \Gscale@div\@tempa{\textheight}{\dimexpr\ht\pandoc@box+\dp\pandoc@box\relax}%
  \Gscale@div\@tempb{\linewidth}{\wd\pandoc@box}%
  \ifdim\@tempb\p@<\@tempa\p@\let\@tempa\@tempb\fi% select the smaller of both
  \ifdim\@tempa\p@<\p@\scalebox{\@tempa}{\usebox\pandoc@box}%
  \else\usebox{\pandoc@box}%
  \fi%
}
% Set default figure placement to htbp
\def\fps@figure{htbp}
\makeatother


% definitions for citeproc citations
\NewDocumentCommand\citeproctext{}{}
\NewDocumentCommand\citeproc{mm}{%
  \begingroup\def\citeproctext{#2}\cite{#1}\endgroup}
\makeatletter
 % allow citations to break across lines
 \let\@cite@ofmt\@firstofone
 % avoid brackets around text for \cite:
 \def\@biblabel#1{}
 \def\@cite#1#2{{#1\if@tempswa , #2\fi}}
\makeatother
\newlength{\cslhangindent}
\setlength{\cslhangindent}{1.5em}
\newlength{\csllabelwidth}
\setlength{\csllabelwidth}{3em}
\newenvironment{CSLReferences}[2] % #1 hanging-indent, #2 entry-spacing
 {\begin{list}{}{%
  \setlength{\itemindent}{0pt}
  \setlength{\leftmargin}{0pt}
  \setlength{\parsep}{0pt}
  % turn on hanging indent if param 1 is 1
  \ifodd #1
   \setlength{\leftmargin}{\cslhangindent}
   \setlength{\itemindent}{-1\cslhangindent}
  \fi
  % set entry spacing
  \setlength{\itemsep}{#2\baselineskip}}}
 {\end{list}}
\usepackage{calc}
\newcommand{\CSLBlock}[1]{\hfill\break\parbox[t]{\linewidth}{\strut\ignorespaces#1\strut}}
\newcommand{\CSLLeftMargin}[1]{\parbox[t]{\csllabelwidth}{\strut#1\strut}}
\newcommand{\CSLRightInline}[1]{\parbox[t]{\linewidth - \csllabelwidth}{\strut#1\strut}}
\newcommand{\CSLIndent}[1]{\hspace{\cslhangindent}#1}



\setlength\parindent{1cm}
\setlength\parskip{0cm}




\usepackage{newtx}

\defaultfontfeatures{Scale=MatchLowercase}
\defaultfontfeatures[\rmfamily]{Ligatures=TeX,Scale=1}





\title{XXX Your Title XXX:}


\shorttitle{}


\usepackage{etoolbox}


\duedate{XXX January 13, 2026 XXX}




\author{XXX Your Name XXX}



\affiliation{
{Cologne, }}




\leftheader{XXX}






\authornote{ 

\par{       }
\par{Correspondence concerning this article should be addressed to XXX
Your Name XXX}
}

\makeatletter
\let\endoldlt\endlongtable
\def\endlongtable{
\hline
\endoldlt
}
\makeatother

\urlstyle{same}



\makeatletter
\@ifpackageloaded{caption}{}{\usepackage{caption}}
\AtBeginDocument{%
\ifdefined\contentsname
  \renewcommand*\contentsname{Table of contents}
\else
  \newcommand\contentsname{Table of contents}
\fi
\ifdefined\listfigurename
  \renewcommand*\listfigurename{List of Figures}
\else
  \newcommand\listfigurename{List of Figures}
\fi
\ifdefined\listtablename
  \renewcommand*\listtablename{List of Tables}
\else
  \newcommand\listtablename{List of Tables}
\fi
\ifdefined\figurename
  \renewcommand*\figurename{Figure}
\else
  \newcommand\figurename{Figure}
\fi
\ifdefined\tablename
  \renewcommand*\tablename{Table}
\else
  \newcommand\tablename{Table}
\fi
}
\@ifpackageloaded{float}{}{\usepackage{float}}
\floatstyle{ruled}
\@ifundefined{c@chapter}{\newfloat{codelisting}{h}{lop}}{\newfloat{codelisting}{h}{lop}[chapter]}
\floatname{codelisting}{Listing}
\newcommand*\listoflistings{\listof{codelisting}{List of Listings}}
\makeatother
\makeatletter
\makeatother
\makeatletter
\@ifpackageloaded{caption}{}{\usepackage{caption}}
\@ifpackageloaded{subcaption}{}{\usepackage{subcaption}}
\makeatother

% From https://tex.stackexchange.com/a/645996/211326
%%% apa7 doesn't want to add appendix section titles in the toc
%%% let's make it do it
\makeatletter
\xpatchcmd{\appendix}
  {\par}
  {\addcontentsline{toc}{section}{\@currentlabelname}\par}
  {}{}
\makeatother

\begin{document}

    \cleardoublepage
\thispagestyle{empty}
\hfill \includegraphics[width=7cm]{logo.png}\\
{\centering
  {XXX Hochschule Fresenius / University of Applied Science XXX\\
XXX Faculty of Economics and Media XXX\\
XXX International Business School XXX\\
XXX International Business Management XXX\\
 XXX Cologne Campus XXX  \par
}
  \hbox{}\vskip 0cm plus 1fill
  {\Large \bfseries XXX Your Title XXX \par}
    \vspace{8ex}
  {XXX Bachelor's/Master's Thesis XXX \par}
    {in partial fulfillment of the requirements for the degree of XXX
Bachelor of Arts (B.A.)/Bachelor of Laws (LL.B.)/Bachelor of Science
(B.Sc.)/Master of Arts (M.A.)/Master of Business Administration
(MBA)/Master of Science (M.Sc.) XXX \par}
      \vfill
      {XXX Your Name XXX \par}
      \vspace{0ex}
  { \par}
  \vspace{0ex}
    {\large  \par}
  \vspace{0ex}
  { \par}
    \vspace{0ex}
    {\large  \par}
  \vspace{0ex}
  { \par}
    {Student ID No.: XXX \par}
      \vspace{8ex}
  {1\textsuperscript{st} examiner: Prof.~Dr.~XXX \par}
  {2\textsuperscript{nd} examiner: Prof.~Dr.~XXX \par}
    \vfill
    {Due Date: XXX January 13, 2026 XXX \par}
          %
  \clearpage
}



\renewcommand{\contentsname}{Table of Contents}

\section[Introduction]{XXX Your Title XXX}

\setcounter{secnumdepth}{5}

\setlength\LTleft{0pt}


\section*{Abstract}\label{abstract}
\addcontentsline{toc}{section}{Abstract}

\noindent  This template was created by Prof.~Dr.~Stephan Huber and is
based on the Quarto extension apaquarto
(\citeproc{ref-Schneider2024quarto}{Schneider, 2024}) and the
\LaTeX~package apa7 (\citeproc{ref-Weiss2022Formatting}{Weiss, 2022}).
For the most part, it is designed in accordance with APA stlye (7th
Edition) guidelines. However, some adjustments have been made to conform
to the formatting requirements specified in the ``Handbook of Academic
Writing'' (\citeproc{ref-Hildebrandt2019Handbook}{Hildebrandt \& Nelke,
2019, Section 4.1.2}).

\newpage
\thispagestyle{empty}
\tableofcontents
\clearpage
\listoffigures
\clearpage
\listoftables

\newpage

\section{Introduction}\label{introduction}

This is a template. You can use it to write your student paper or your
thesis with Quarto. The formatting follows the guidelines set out in
American Psychological Association
(\citeproc{ref-ConciseGuideAPA2020}{2020}), also known as the APA rules
in the 7th Edition. This template was written by me, Prof.~Dr.~Stephan
Huber\footnote{Email: \texttt{stephan.huber@hs-fresenius.de}}. I have
adapted the Quarto extension \texttt{apaquarto} from Schneider
(\citeproc{ref-Schneider2024quarto}{2024}) and the \LaTeX~package apa7
(\citeproc{ref-Weiss2022Formatting}{Weiss, 2022}). The title page was
created according to the specifications of the University of Applied
Science Fresenius. In comparison to APA 7 a few little things differ.
For example, the headings have different font sizes and the line spacing
is 1.5.\\
If you have any suggestions for improvement, please let me know. If you
need help with Quarto, you are welcome to drop by during my office
hours. All informations on how to contact me can be found on my
\href{https://hubchev.github.io/}{personal website}.

You can download the template with the corresponding files on studynet
and here: \url{https://github.com/hubchev/temp_apa_en}.

In the following sections, I will show you a little about how to use
Quarto to write and format text. If you need more information about
Quarto, you can get a lot of information from Schneider
(\citeproc{ref-Schneider2024quarto}{2024}) and the website
\url{https://quarto.org/}.

\section{Formatting Text}\label{formatting-text}

\subsection{Set Sections and Format
Text}\label{set-sections-and-format-text}

APA strictly regulates the formatting of headings, see:
\url{https://apastyle.apa.org/style-grammar-guidelines/paper-format/headings}

\subsubsection{This Is a Subsection}\label{this-is-a-subsection}

A section should not come alone, so\ldots{}

\subsubsection{The Second Subsection}\label{the-second-subsection}

This is where your text belongs.

\subsection{Bold and Italics}\label{bold-and-italics}

\textbf{This is bold text}

\textbf{This is bold text}

\emph{This is italic text}

\emph{This is italic text}

\subsection{Lists}\label{lists}

\begin{itemize}
\tightlist
\item
  Create a list by starting a line with \texttt{+}, \texttt{-} or
  \texttt{*}
\item
  Sublists are created by indenting by 2 spaces:

  \begin{itemize}
  \tightlist
  \item
    Changing the selection character forces the start of a new list:

    \begin{itemize}
    \tightlist
    \item
      bli
    \item
      bla
    \item
      blubb
    \end{itemize}
  \end{itemize}
\item
  Very simple!
\end{itemize}

Numbered lists are simple:

\begin{enumerate}
\def\labelenumi{\arabic{enumi}.}
\item
  one argument.
\item
  another argument.
\item
  the best argument.
\item
  you can use consecutive numbers\ldots{}
\item
  \ldots or keep all numbers as \texttt{1.}
\end{enumerate}

You can also start numbering with a higher number:

\begin{enumerate}
\def\labelenumi{\arabic{enumi}.}
\setcounter{enumi}{56}
\tightlist
\item
  foo
\item
  bar
\end{enumerate}

\subsection{Formulas}\label{formulas}

If \(a \ne 0\), there are two solutions to the equation
\((ax^2 + bx + c = 0)\) and they are
\[ x = \frac{-b \pm \sqrt{b^2-4ac}}{2a} \]

\subsection{\texorpdfstring{\href{https://github.com/markdown-it/markdown-it-footnote}{Footnotes}}{Footnotes}}\label{footnotes}

Footnote 1 Reference\footnote{Footnote \textbf{may contain markup}

  and multiple paragraphs.}.

Footnote 2 Reference \footnote{Footnote text.}.

Inline footnote\footnote{text of inline footnote} Definition.

Double footnote reference\footnote{Footnote text.}.

\subsection{Tables}\label{tables}

Tables can be created flexibly with Markdown. The possibilities are
explained on the website
\url{https://quarto.org/docs/authoring/tables.html}.
Table~\ref{tbl-example}, for example, is created with Markdown code.

\begin{table}

{\caption{{An Example Table Created With
Markdown}{\label{tbl-example}}}}

\addtocounter{table}{-1}

\begin{longtable}[]{@{}llrc@{}}
\toprule\noalign{}
Default & Left & Right & Center \\
\midrule\noalign{}
\endhead
\bottomrule\noalign{}
\endlastfoot
12 & 12 & 12 & 12 \\
123 & 123 & 123 & 123 \\
1 & 1 & 1 & 1 \\
\end{longtable}

\noindent \emph{Note.} Here is a note.

\end{table}

\subsection{Figures}\label{figures}

\subsubsection{Include Image Files}\label{sec-imageload}

In Figure~\ref{fig-logo} you can see the logo of the university. In
Figure~\ref{fig-logo2}, the logo is displayed smaller and integrated
using a different method. However, both methods are practically
equivalent. Finally, the tiny logo in Figure~\ref{fig-logo3} offers
another method of integrating images.

\begin{figure}[t]

{\caption{{A Large Logo}{\label{fig-logo}}}}

\includegraphics[width=3in,height=\textheight,keepaspectratio]{logo.png}

\noindent \emph{Note.} Here is a note about the image.

\end{figure}

\begin{figure}[t]

\caption{\label{fig-logo2}A Medium Sized Logo.}

\includegraphics[width=0.3\linewidth,height=\textheight,keepaspectratio]{logo.png}

\emph{Note.} There is also a note about the image here.

\end{figure}%

\begin{figure}[t]

{\caption{{The Tiny Logo of the University.}{\label{fig-logo3}}}}

\includegraphics[width=4in,height=\textheight,keepaspectratio]{logo.png}

\noindent \emph{Note.} There is also a note about the image here.

\end{figure}

\subsubsection{Include Graphics Created With
R}\label{include-graphics-created-with-r}

In Figure~\ref{fig-plotcar} a scatterplot is visualized which was
generated directly in R. This has the advantage that the data is
generated and visualized directly in Quarto. The data is therefore
always up-to-date, changes can be made directly here, the work is
completely transparent and replicable. In addition, there is no need to
save and export the graphic.

In the Appendix \ref{sec-Ap1}, you find another exampl, that is,
Figure~\ref{fig-logo4}. Appendix \ref{sec-Ap2} contains
Figure~\ref{fig-logo5}.

\begin{figure}[t]

{\caption{{This Is a Caption for an Ugly Image.}{\label{fig-plotcar}}}}

\pandocbounded{\includegraphics[keepaspectratio]{roadshow_apa_en_files/figure-pdf/fig-plotcar-1.pdf}}

\noindent \emph{Note.} There is also a note about the image here.

\end{figure}

\section{Sources}\label{sources}

\subsection{BibTeX}\label{bibtex}

This template comes with the file \texttt{literatur.bib}. This is a
BibTeX file and makes it easier to cite sources and create
bibliographies. Here is an explanation of how a BibTeX file works and
why it is useful.

A BibTeX file makes it possible to store and organize all references in
one place. This makes it easier to manage sources, especially for large
works. Quarto can automatically access the BibTeX file to create
citations and bibliographies. This saves time and reduces errors
compared to creating bibliographies manually. By using a BibTeX file,
citations and bibliographies are formatted consistently, according to
the specifications of the respective citation style. I recommend the use
of a reference management program. You can find more information on this
in Section~\ref{sec-jabref}.

A BibTeX file is a text-based file with the extension \texttt{.bib} that
contains bibliographic entries. Each entry in the file describes a
source (e.g.~a book, an article, a website) and contains various fields
such as author, title, year and publisher. As an example, I have packed
all possible partly fictitious entries into the file. Here are the first
lines of the file:

\begin{Shaded}
\begin{Highlighting}[]
\NormalTok{@Article\{Huber2016,}
\NormalTok{  author = \{Stephan Huber and Christoph Rust\},}
\NormalTok{  title = \{osrmtime: Calculate Travel Time and Distance with \{OpenStreetMap\} Data Using the \{Open Source Routing Machine\} (\{OSRM\})\},}
\NormalTok{  journal = \{The Stata Journal\},}
\NormalTok{  year = \{2016\},}
\NormalTok{  volume = \{16\},}
\NormalTok{  number = \{2\},}
\NormalTok{  pages = \{416{-}423\}}
\NormalTok{\}}
\end{Highlighting}
\end{Shaded}

Each literature entry has a similar structure. First comes the
specification of the type of literature (here: \texttt{@Article}). Then
the so-called BibTeX key (here: \texttt{Huber2016}). This makes it
possible to set a corresponding reference in the text, see
Section~\ref{sec-cite}. Finally, there is the information that is
further processed by the selected citation style (default: APA 7). In
Appendix \ref{sec-bibtexfile} you find the BibTeX entries for various
types of documents.

\subsection{Literature Management}\label{sec-jabref}

A BibTeX file can quickly become confusing. I therefore recommend using
a reference management program such as JabRef, see:
\url{www.jabref.org}. The program is free and works on all your devices
and with all operating systems. In particular, it can handle BibTeX
files well. You may wonder what a BibTeX file is. Install JabRef and
open the \texttt{bibliography.bib} file of this template, then you will
understand what I mean.

\subsection{APA, Chicago, and CSL}\label{sec-apa}

The predefined bibliography style of this template is APA in the 7th
edition. You can change this. To do this, simply specify the
corresponding style in the (YAML) header of this template. This is done
using a file with the extension \texttt{CSL} which stands for ``Citation
Style Language''. This file contains the citation rules so that the
computer can implement them. On \url{https://www.zotero.org/styles} you
will find a variety of styles. For example, if you want to use APA in
the 6th edition, enter the following in the header:

\begin{Shaded}
\begin{Highlighting}[]
\NormalTok{csl: "https://www.zotero.org/styles/apa{-}6th{-}edition"}
\end{Highlighting}
\end{Shaded}

If you want to use Chicago Manual Stlye in the 17th edition, enter the
following in the header:

\begin{Shaded}
\begin{Highlighting}[]
\NormalTok{csl: "https://www.zotero.org/styles/chicago{-}author{-}date" }
\end{Highlighting}
\end{Shaded}

\subsection{Cite Literature}\label{sec-cite}

Other works are often cited in scientific texts. This can be done in
many different ways. Table~\ref{tbl-letters} shows how to cite.

\begin{table}

{\caption{{This Is How Literature Can Be Cited}{\label{tbl-letters}}}}

\addtocounter{table}{-1}

\begin{longtable}[]{@{}
  >{\raggedright\arraybackslash}p{(\linewidth - 2\tabcolsep) * \real{0.4545}}
  >{\raggedright\arraybackslash}p{(\linewidth - 2\tabcolsep) * \real{0.5455}}@{}}
\toprule\noalign{}
\begin{minipage}[b]{\linewidth}\raggedright
Code
\end{minipage} & \begin{minipage}[b]{\linewidth}\raggedright
How it appears in the text
\end{minipage} \\
\midrule\noalign{}
\endhead
\bottomrule\noalign{}
\endlastfoot
\texttt{@Huber2016} & Huber and Rust (\citeproc{ref-Huber2016}{2016}) \\
\texttt{@Huber2016{[}2{]}} & Huber and Rust
(\citeproc{ref-Huber2016}{2016, p. 2}) \\
\texttt{{[}@Huber2016{]}} & (\citeproc{ref-Huber2016}{Huber \& Rust,
2016}) \\
\texttt{{[}@Huber2016,\ 3-5{]}} & (\citeproc{ref-Huber2016}{Huber \&
Rust, 2016, pp. 3--5}) \\
\texttt{{[}@Huber2016;\ @Aust2023{]}} & (\citeproc{ref-Aust2023}{Aust \&
Barth, 2023}; \citeproc{ref-Huber2016}{Huber \& Rust, 2016}) \\
\texttt{{[}see\ @Aust2023,\ p.\ 31;\ @Huber2016,\ p.\ 13{]}} & (see
\citeproc{ref-Aust2023}{Aust \& Barth, 2023, p. 31};
\citeproc{ref-Huber2016}{Huber \& Rust, 2016, p. 13}) \\
\end{longtable}

\end{table}

Hyperlinks can also be set. For example,
\texttt{{[}Google{]}(www.google.de)} appears as
\href{www.google.de}{Google}. In academic papers, sources should always
be included in the bibliography and hyperlinks do not work in printed
form. Here is an example: \href{www.google.de}{Google} is a popular
online search engine (see \citeproc{ref-Google2023Google}{Google,
2023}).

\subsection{Various Formats}

Here are some fictitious examples of different types of literature:
Unpublished (\citeproc{ref-unpublished}{1993}), WorkingPaper and
Arbeitsname (\citeproc{ref-techreport}{1993}), PHD
(\citeproc{ref-phdthesis}{2011}), Misc (\citeproc{ref-misc}{1993}),
Masterthesis (\citeproc{ref-mastersthesis}{2004}), Editor
(\citeproc{ref-incollection}{2022}), Inbook
(\citeproc{ref-inbook}{1993}), Conference
(\citeproc{ref-conference}{1986}), Booklet
(\citeproc{ref-booklet}{1996}), Book (\citeproc{ref-book}{2023}),
Article (\citeproc{ref-article}{2011}), Proceedings
(\citeproc{ref-proceedings}{1993}), Website
(\citeproc{ref-webpage}{2023}).

The corresponding bibtex file should contain the entries. They are
displayed in Section~\ref{sec-bibtexfile}.

\clearpage

\section*{References}\label{references}
\addcontentsline{toc}{section}{References}

\phantomsection\label{refs}
\begin{CSLReferences}{1}{0}
\bibitem[\citeproctext]{ref-ConciseGuideAPA2020}
American Psychological Association. (2020). \emph{Concise guide to {APA
Style}: {The} official {APA Style} guide for students (7th ed.).}
\url{https://doi.org/10.1037/0000173-000}

\bibitem[\citeproctext]{ref-article}
Article, S. (2011). The title of the work. \emph{The Name of the
Journal}, \emph{4}(2), 201--213.

\bibitem[\citeproctext]{ref-Aust2023}
Aust, F., \& Barth, M. (2023). \emph{{papaja}: {Prepare} reproducible
{APA} journal articles with {R} markdown}.
\url{https://github.com/crsh/papaja}

\bibitem[\citeproctext]{ref-book}
Book, D. (2023). \emph{The title of the work} (3rd ed., Vol. 4). The
name of the publisher.

\bibitem[\citeproctext]{ref-booklet}
Booklet, B. (1996, July). \emph{The title of the work}. How it was
published.

\bibitem[\citeproctext]{ref-conference}
Conference, D. (1986). \emph{The title of the work} (N. of the editor,
Ed.; 2; Vol. 4, p. 213). The organization; The publisher.

\bibitem[\citeproctext]{ref-incollection}
Editor, E. (2022). The title of the work. In T. editor (Ed.), \emph{The
title of the book} (3rd ed., Vol. 4, pp. 101--114). The name of the
publisher.

\bibitem[\citeproctext]{ref-Google2023Google}
Google. (2023). \emph{Google search}. January 30, 2023.
\url{https://www.google.com/}

\bibitem[\citeproctext]{ref-Hildebrandt2019Handbook}
Hildebrandt, J., \& Nelke, M. (Eds.). (2019). \emph{Handbook of academic
writing}. VNR Verlag f{ü}r die Deutsche Wirtschaft.

\bibitem[\citeproctext]{ref-Huber2016}
Huber, S., \& Rust, C. (2016). Osrmtime: Calculate travel time and
distance with {OpenStreetMap} data using the {Open Source Routing
Machine} ({OSRM}). \emph{The Stata Journal}, \emph{16}(2), 416--423.

\bibitem[\citeproctext]{ref-inbook}
Inbook, S. (1993). \emph{The title of the work} (3rd ed., Vol. 4, pp.
201--213). The name of the publisher.

\bibitem[\citeproctext]{ref-mastersthesis}
Masterthesis, A. (2004). \emph{The title of the work} {[}Master's
thesis{]}. The school of the thesis.

\bibitem[\citeproctext]{ref-misc}
Misc, A. (1993). \emph{The title of the work}. How it was published.

\bibitem[\citeproctext]{ref-phdthesis}
PHD, A. (2011). \emph{The title of the work} {[}PhD thesis{]}. The
school of the thesis.

\bibitem[\citeproctext]{ref-proceedings}
Proceedings, L. (Ed.). (1993). \emph{The title of the work} (Vol. 4).
The organization; The name of the publisher.

\bibitem[\citeproctext]{ref-Schneider2024quarto}
Schneider, W. J. (2024). \emph{A quarto extension for creating APA7
documents in .docx, .html, and .pdf formats}.
\url{https://wjschne.github.io/apaquarto/}

\bibitem[\citeproctext]{ref-unpublished}
Unpublished, A. (1993). \emph{The title of the work}.

\bibitem[\citeproctext]{ref-webpage}
Website, A. (2023). \emph{Website title}. \url{http://website-url.com}

\bibitem[\citeproctext]{ref-Weiss2022Formatting}
Weiss, D. A. (2022). \emph{Formatting documents in {APA} style (7th
edition) with the apa7 {LATEX} class} (Version 2.16 2022-07-25).
\url{https://ctan.org/pkg/apa7}

\bibitem[\citeproctext]{ref-techreport}
WorkingPaper, P., \& Arbeitsname, S. (1993). \emph{The title of the
work} (2). The institution that published.

\end{CSLReferences}

\clearpage
\appendix

\section{Additional Figures}\label{sec-Ap1}

Here is text and Figure~\ref{fig-logo4}.

\begin{figure}[h]

{\caption{{Logo of the University.}{\label{fig-logo4}}}}

\includegraphics[width=2in,height=\textheight,keepaspectratio]{logo.png}

\noindent \emph{Note.} This is the logo of Hochschule Fresenius.

\end{figure}

\newpage

\section{Second Attachment}\label{sec-Ap2}

Here is text and Figure~\ref{fig-logo5}.

\begin{figure}[h]

{\caption{{The Logo Again}{\label{fig-logo5}}}}

\includegraphics[width=2in,height=\textheight,keepaspectratio]{logo.png}

\noindent \emph{Note.} This is the logo of Hochschule Fresenius.

\end{figure}

\newpage

\section{BibTex file}\label{sec-bibtexfile}

\begin{Shaded}
\begin{Highlighting}[]
\NormalTok{    @Unpublished\{unpublished,}
\NormalTok{        author = \{Andy Unpublished\},}
\NormalTok{        title  = \{The title of the work\},}
\NormalTok{        month  = \{7\},}
\NormalTok{        year   = \{1993\},}
\NormalTok{    \}}
    
\NormalTok{    @TechReport\{techreport,}
\NormalTok{        author      = \{Peter WorkingPaper\},}
\NormalTok{        title       = \{The title of the work\},}
\NormalTok{        institution = \{The institution that published\},}
\NormalTok{        year        = \{1993\},}
\NormalTok{        number      = \{2\},}
\NormalTok{        address     = \{The address of the publisher\},}
\NormalTok{        month       = \{7\},}
\NormalTok{    \}}
    
\NormalTok{    @PhdThesis\{phdthesis,}
\NormalTok{        author  = \{Andreas PHD\},}
\NormalTok{        title   = \{The title of the work\},}
\NormalTok{        school  = \{The school of the thesis\},}
\NormalTok{        year    = \{2011\},}
\NormalTok{        address = \{The address of the publisher\},}
\NormalTok{        month   = \{7\},}
\NormalTok{    \}}
    
\NormalTok{    @Misc\{misc,}
\NormalTok{        author       = \{Alexandra Misc\},}
\NormalTok{        title        = \{The title of the work\},}
\NormalTok{        howpublished = \{How it was published\},}
\NormalTok{        month        = \{4\},}
\NormalTok{        year         = \{1993\},}
\NormalTok{    \}}
    
\NormalTok{    @MastersThesis\{mastersthesis,}
\NormalTok{        author  = \{Alex Masterthesis\},}
\NormalTok{        title   = \{The title of the work\},}
\NormalTok{        school  = \{The school of the thesis\},}
\NormalTok{        year    = \{2004\},}
\NormalTok{        address = \{The address of the publisher\},}
\NormalTok{        month   = \{10\},}
\NormalTok{    \}}
    
\NormalTok{    @InCollection\{incollection,}
\NormalTok{        author    = \{Eli Editor\},}
\NormalTok{        title     = \{The title of the work\},}
\NormalTok{        booktitle = \{The title of the book\},}
\NormalTok{        publisher = \{The name of the publisher\},}
\NormalTok{        year      = \{2022\},}
\NormalTok{        editor    = \{The editor\},}
\NormalTok{        volume    = \{4\},}
\NormalTok{        series    = \{2\},}
\NormalTok{        chapter   = \{8\},}
\NormalTok{        pages     = \{101{-}114\},}
\NormalTok{        address   = \{The address of the publisher\},}
\NormalTok{        edition   = \{3\},}
\NormalTok{        month     = \{7\},}
\NormalTok{    \}}
    
\NormalTok{    @InBook\{inbook,}
\NormalTok{        chapter   = \{8\},}
\NormalTok{        pages     = \{201{-}213\},}
\NormalTok{        title     = \{The title of the work\},}
\NormalTok{        publisher = \{The name of the publisher\},}
\NormalTok{        year      = \{1993\},}
\NormalTok{        author    = \{Suzy Inbook\},}
\NormalTok{        volume    = \{4\},}
\NormalTok{        series    = \{5\},}
\NormalTok{        address   = \{The address of the publisher\},}
\NormalTok{        edition   = \{3\},}
\NormalTok{        month     = \{7\},}
\NormalTok{    \}}
    
\NormalTok{    @Conference\{conference,}
\NormalTok{        author       = \{David Conference\},}
\NormalTok{        title        = \{The title of the work\},}
\NormalTok{        booktitle    = \{The title of the book\},}
\NormalTok{        year         = \{1986\},}
\NormalTok{        editor       = \{The editor\},}
\NormalTok{        volume       = \{4\},}
\NormalTok{        series       = \{5\},}
\NormalTok{        pages        = \{213\},}
\NormalTok{        address      = \{The address of the publisher\},}
\NormalTok{        month        = \{7\},}
\NormalTok{        organization = \{The organization\},}
\NormalTok{        publisher    = \{The publisher\},}
\NormalTok{    \}}
    
\NormalTok{    @Booklet\{booklet,}
\NormalTok{        title        = \{The title of the work\},}
\NormalTok{        author       = \{Betty Booklet\},}
\NormalTok{        howpublished = \{How it was published\},}
\NormalTok{        address      = \{The address of the publisher\},}
\NormalTok{        month        = \{7\},}
\NormalTok{        year         = \{1996\},}
\NormalTok{    \}}
    
\NormalTok{    @Book\{book,}
\NormalTok{        title     = \{The title of the work\},}
\NormalTok{        publisher = \{The name of the publisher\},}
\NormalTok{        year      = \{2023\},}
\NormalTok{        author    = \{Debbie Book\},}
\NormalTok{        volume    = \{4\},}
\NormalTok{        series    = \{10\},}
\NormalTok{        address   = \{The address\},}
\NormalTok{        edition   = \{3\},}
\NormalTok{        month     = \{7\},}
\NormalTok{        isbn      = \{3257442892\},}
\NormalTok{    \}}
    
\NormalTok{    @Article\{article,}
\NormalTok{        author  = \{Sepp Article\},}
\NormalTok{        title   = \{The title of the work\},}
\NormalTok{        journal = \{The name of the journal\},}
\NormalTok{        year    = \{2011\},}
\NormalTok{        volume  = \{4\},}
\NormalTok{        number  = \{2\},}
\NormalTok{        pages   = \{201{-}213\},}
\NormalTok{        month   = \{7\},}
\NormalTok{    \}}
    
\NormalTok{    @Proceedings\{proceedings,}
\NormalTok{        title        = \{The title of the work\},}
\NormalTok{        year         = \{1993\},}
\NormalTok{        editor       = \{Luc Proceedings\},}
\NormalTok{        volume       = \{4\},}
\NormalTok{        series       = \{5\},}
\NormalTok{        address      = \{The address of the publisher\},}
\NormalTok{        publisher    = \{The name of the publisher\},}
\NormalTok{        month        = \{7\},}
\NormalTok{        organization = \{The organization\},}
\NormalTok{    \}}
    
\NormalTok{    @Misc\{webpage,}
\NormalTok{        author = \{Author Website\},}
\NormalTok{        title  = \{Website Title\},}
\NormalTok{        year   = \{2023\},}
\NormalTok{        note   = \{Accessed on April 14, 2023\},}
\NormalTok{        url    = \{http://website{-}url.com\},}
\NormalTok{    \}}
\end{Highlighting}
\end{Shaded}







\end{document}
